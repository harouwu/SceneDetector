% You should title the file with a .tex extension (hw1.tex, for example)
\documentclass[11pt]{article}

\usepackage{amsmath}
\usepackage{amssymb}
\usepackage{amsthm}
\usepackage{fancyhdr}
\usepackage{algorithm}
\usepackage{algorithmic}
\usepackage{graphicx}
\usepackage{hyperref}
\usepackage{listings}
\usepackage{color}
 
\definecolor{codegreen}{rgb}{0,0.6,0}
\definecolor{codegray}{rgb}{0.5,0.5,0.5}
\definecolor{codepurple}{rgb}{0.58,0,0.82}
\definecolor{backcolour}{rgb}{0.95,0.95,0.92}
 
\lstdefinestyle{mystyle}{
    backgroundcolor=\color{backcolour},   
    commentstyle=\color{codegreen},
    keywordstyle=\color{magenta},
    numberstyle=\tiny\color{codegray},
    stringstyle=\color{codepurple},
    basicstyle=\footnotesize,
    breakatwhitespace=false,         
    breaklines=true,                 
    captionpos=b,                    
    keepspaces=true,                 
    numbers=left,                    
    numbersep=5pt,                  
    showspaces=false,                
    showstringspaces=false,
    showtabs=false,                  
    tabsize=2
}
 
\lstset{style=mystyle}

\oddsidemargin0cm
\topmargin-2cm     %I recommend adding these three lines to increase the 
\textwidth16.5cm   %amount of usable space on the page (and save trees)
\textheight23.5cm  

\theoremstyle{theorem}
\newtheorem{theorem}{Theorem}[section]
\theoremstyle{lemma}
\newtheorem{lemma}{Lemma}[section]
\theoremstyle{property}
\newtheorem{property}{Property}[section]
\theoremstyle{definition}
\newtheorem{definition}{Definition}[section]
\theoremstyle{assumption}
\newtheorem{assumption}{Assumption}[section]


\newcommand{\question}[2] {\vspace{.25in} \hrule\vspace{0.5em}
\noindent{\bf #1: #2} \vspace{0.5em}
\hrule \vspace{.10in}}
\renewcommand{\part}[1] {\vspace{.10in} {\bf (#1)}}

\newcommand{\myname}{Yiming Wu}
\newcommand{\myandrew}{wyiming@andrew.cmu.edu}
\newcommand{\myhwnum}{1}

\setlength{\parindent}{0pt}
\setlength{\parskip}{5pt plus 1pt}
 
\pagestyle{fancyplain}
\lhead{\fancyplain{}{\textbf{HW\myhwnum}}}      % Note the different brackets!
\rhead{\fancyplain{}{\myname\\ \myandrew}}
\chead{\fancyplain{}{16-720 Computer Vision}}

\begin{document}

\medskip                        % Skip a "medium" amount of space
                                % (latex determines what medium is)
                                % Also try: \bigskip, \littleskip

\thispagestyle{plain}
\begin{center}                  % Center the following lines
{\Large 16-720 Assignment \myhwnum} \\
\myname \\
\myandrew \\
Computer Vision\\
2015.09.21 \\
\end{center}

\question{1}{What properties do each of the filter functions pick up?}
\part{a} We can look into all the filters.

\begin{lstlisting}[language=Matlab, caption=Python example]
for scale = gaussianScales
    idx = idx + 1;
    filterBank{idx} = fspecial('gaussian', 2*ceil(scale*2.5)+1, scale);
end

for scale = logScales
    idx = idx + 1;
    filterBank{idx} = fspecial('log', 2*ceil(scale*2.5)+1, scale);
end

for scale = dxScales
    idx = idx + 1;
    f = fspecial('gaussian', 2*ceil(scale*2.5) + 1, scale);
    f = imfilter(f, [-1 0 1], 'same');
    filterBank{idx} = f;
end

for scale = dyScales
    idx = idx + 1;
    f = fspecial('gaussian', 2*ceil(scale*2.5) + 1, scale);
    f = imfilter(f, [-1 0 1]', 'same');
    filterBank{idx} = f;
end
\end{lstlisting}

\begin{enumerate}
	\item The first 5 filters are all Gaussian filters(Gaussian lowpass filter) with scale from 1 to 16. Mainly they let low frequence signals pass, which blur the object and emphasize on the scene.
	\item The second 5 filters are log(Laplacian of Gaussian filter) filters. They detect edges.
	\item The third 5 filters are derivatives of Gaussian in x direction, which will detect edges in y direction mainly.
	\item The last 5 filters are derivatives of Gaussian in y direction, which will detect edges in x direction mainly.
\end{enumerate}


\newpage

\question{2}{Quantitative Evaluation}
\part{a} My $evaluateRecognitionSystem.m$.
\begin{lstlisting}[language=Matlab, caption=My evaluateRecognitionSystem.m]
load('../dat/traintest.mat','test_imagenames','test_labels','mapping');
load('vision.mat');
classNum = 8;
C = zeros(classNum);
testSize = size(test_labels, 2);
for k = 1:testSize
    i = test_labels(k);
    myGuess = guessImage(['../dat/', test_imagenames{k}]);
    index = strfind(mapping, myGuess);
    
    j = find(not(cellfun('isempty', index)));
    
    C(i,j) = C(i,j) + 1;
end

rate = trace(C) / sum(C(:)) * 100;
fprintf('accuracy=%d%',rate);
\end{lstlisting}


\end{document}

